\RequirePackage[]{silence}
\WarningFilter{latex}{Marginpar on page}
\WarningFilter{lcg}{Using an already existing counter rand}
\documentclass[article, oneside]{aaltoseries}
\usepackage[utf8]{inputenc}
\usepackage[english]{babel}
\usepackage{IEEEtrantools}
\usepackage[hyphens]{url} % allow hyphens to break urls
\Urlmuskip=0mu  plus 1mu % allow for some spacing inside urls
\usepackage[hidelinks]{hyperref}
\usepackage[acronym]{glossaries}
\usepackage{amsfonts}
\usepackage{todonotes}
\usepackage[normalem]{ulem}
\usepackage{xcolor}
\usepackage{xspace}
\usepackage{textcomp}
\usepackage{etoolbox}
\usepackage[numbers]{natbib}
\newcommand{\TODO}[1]{\todo[inline]{#1}}
\newcommand{\jwnote}[1]{\todo[size=\small,color=blue!40]{#1}}
\newcommand{\JWNOTE}[1]{\todo[size=\small,inline,color=blue!40]{#1}}
\newcommand{\modif}[1]{\textcolor{blue}{#1}}
\newcommand\remove{\bgroup\markoverwith
{\textcolor{red}{\rule[0.5ex]{2pt}{0.8pt}}}\ULon}
\newcommand{\discuss}[1]{\todo[size=\small,color=green!40]{#1}}

% \geometry{left=4cm,right=4cm,marginparwidth=3.5cm,marginparsep=5mm} % TODO: remove
\geometry{left=1cm,right=7cm,marginparwidth=6.5cm,marginparsep=5mm} % TODO: remove
% \reversemarginpar % for todo notes on the wider side

% Use “\tocite{}” to get “citation needed” in document
\newcommand\tocite[1]{
	\todo{\ifstrempty{#1}{Citation needed}{#1}}
	[\,\textbf{\color{red}!}\,]
}


% ************************ Custom Referencing **********************************
\newcommand{\Fref}[1]{Figure~\ref{#1}}
\newcommand{\fref}[1]{figure~\ref{#1}}
\newcommand{\Tref}[1]{Table~\ref{#1}}
\newcommand{\tref}[1]{table~\ref{#1}}
\newcommand{\Eref}[1]{Equation~\ref{#1}}
\newcommand{\eref}[1]{equation~\ref{#1}}
\newcommand{\Sref}[1]{Section~\ref{#1}}
\newcommand{\sref}[1]{section~\ref{#1}}
\newcommand{\Lref}[1]{Listing~\ref{#1}}
\newcommand{\lref}[1]{listing~\ref{#1}}


\newcommand{\app}[1]{A\textsc{pp}$_{#1}$\xspace}

\newacronym{os}{OS}{operating system}
\newacronym{aosp}{AOSP}{Android Open Source Project}

\title{Android App Collusion}


\author{Ervin Oro% Your first and last name: do _not_ add your student number
\\\textnormal{\texttt{ervin.oro@aalto.fi}}} % Your Aalto e-mail address
\affiliation{\textbf{Tutor}: Jorden Whitefield} % First and last name of your tutor

%==========================================================

\begin{document}
\bstctlcite{BSTcontrol}

\maketitle
\todo{set proper margins and document format}
%==========================================================

\begin{abstract}
\TODO{abstract}

\vspace{3mm}
\noindent KEYWORDS: \todo[inline, inlinewidth=5cm, noinlinepar]{list, of, key, words}

\end{abstract}

%============================================================

\section{Introduction}
\label{sec:intro}

Android is an \gls{os} that is primarily designed for \remove{mobile devices}\jwnote{smartphones and tablets}. With more than two billion active devices~\cite{AOSP2018}, it is estimated to be the most widely used \gls{os}, surpassing \remove{even} Microsoft Windows~\cite{AWSLLC2018, StatCounter2018}. Android is designed to be an open platform: developed and maintained by Google LLC, but largely released as the \gls{aosp} for everyone to study and \remove{build upon}\jwnote{develop} \cite{AOSP2019c}. Android \gls{os} includes support for apps, which are easily installable application packages that can extend the functionality of devices. Apps can be developed and distributed by anyone with very low barrier of entry.

While this popularity of Android is not reflected by the proportion of malware attacks, most of which still target Windows, both the number and complexity of attacks against Android are increasing~\cite{AVTESTGH2018}. This is especially troublesome as many people increasingly rely on their smartphones -- often to store their personal data, online account credentials, money, and more. \citeauthor{McAfee2018} estimates that revenues for mobile malware authors could be in the billion-dollar range by 2020 \cite{McAfee2018}.

Given the increasing potential damage from Android malware, defending against it is an active area of research. Android uses a multi-layer security approach, combining machine learning, platform security and secure hardware~\cite{AOSP2018}. Machine learning methods are utilised by Google Play Store\discuss{does "Play Protect" also refer to the Play Store checks?} to prevent uploading potentially harmful applications, and by Google Play Protect\footnote{\url{https://www.android.com/play-protect/}}\discuss{citation vs footnote} to scan apps locally on users' devices. Android's platform security has been enhanced over the years with features like\jwnote{the addition of features, for example,} SELinux protections\footnote{\url{https://source.android.com/security/selinux/}}, exploit mitigations\footnote{\url{https://lwn.net/Articles/695991/}}, privilege reductions\footnote{\url{https://android-developers.googleblog.com/2017/07/seccomp-filter-in-android-o.html}}, and encryption. Recent versions of Android leverage hardware security features, including keystore and remote key attestation\footnote{\url{https://android-developers.googleblog.com/2017/09/keystore-key-attestation.html}}, and receive regular software updates. These security mechanisms have been partially successful, as exploit pricing and difficulty are growing by some estimates~\cite{AOSP2018}.

However, malicious actors are continuously developing exploits to bypass existing protections, and a number of threats, e.g., app collusion, cannot yet be reliably detected nor defended against. App collusion is a secret collaboration between apps with malicious intent  (\Sref{sec:def}). This can be facilitated by any of the numerous ways for apps to communicate with each other that the Android system provides (\Sref{sec:methods}). Methods for apps to collude also exist on the iOS platform \cite{Deshotels2016}. Given a malicious app that would be detected and blocked with state of the art security systems, its functionality can be split into several apps, so that each of them would be categorised as benign when analysed separately~\cite{Chen2018}.

Android app collusion is not a new concept~\cite{Schlegel2011}, and multiple attempts have been made to develop a suitable detection system. \TODO{brief overview of existing approaches based on \Sref{sec:approaches}}
% Filtering
% 	rule-based
% 	machine learning
% model based
% runtime

Despite this, there are currently no robust and usable ways to detect app collusions. Most proposed solutions have a large number of false positives due to their inability to differentiate collusion from legitimate collaboration. Furthermore, since the number of possible combinations is exponential in the number of apps, that is, $N^N$\jwnote{What is $N$?}, most proposed solutions apply very aggressive filtering, causing only some malicious combinations to be included into analysis, and others to be reported as false negatives. \remove{Finally, }approaches attempting to overcome both of these issues have been computationally infeasible thus far. Therefore, app collusion remains an open research challenge.

This report aims to provide an overview of app collusion on the Android platform as follows. \Sref{sec:def} discusses the nature of app collusions in general,~\Sref{sec:methods} provides specific overview of methods that can be used for colluding on Android,~\Sref{sec:examples} describes known examples of colluding apps, and~\Sref{sec:approaches} gives a more in-depth systematic overview of approaches that have been taken to collusion detection and their limitations.

\section{Description and definition of app collusion}
\label{sec:def}

The Oxford English Dictionary defines collusion as ``Secret agreement or understanding for purposes of trickery or fraud; underhand scheming or working with another; deceit, fraud, trickery''~\cite{OEDcollusion}. \citeauthor{Asavoae2017}~\cite{Asavoae2017} define collusion for the case of Android apps as the situation where several apps are working together in performing a threat. According to these definitions, app collusion must have the following three properties:

\begin{enumerate}
	
	\item Colluding apps must be working together secretly. Conversely, apps working together in collaboration is common and encouraged practice when such collaboration is well documented \tocite{}.

	\item All colluding apps must be in agreement. A distinctly different but related concept is the ``confused deputy'' attack, where one app mistakenly exposes itself to other installed apps \tocite{Hardy, N.: The confused deputy:(or why capabilities might have been invented. ACMSIGOPS Operating Systems Review 22(4), 36–38 (1988)}.

	\item Colluding apps must have malicious intent. The intent of Android app collusion would then be to violate one of Android's security goals, which are defined in \cite{AOSP2019} as:
	\jwnote{I also wonder if it is worth mentioning OWASP Mobile top 10	security... \url{https://www.owasp.org/index.php/Mobile\_Top\_10\_2016-Top\_10}}

	\begin{singleenums}
		\item to protect app data, user data, and system resources (including the network),
		\item to provide app isolation from the system, other apps, and the user.
	\end{singleenums}
\end{enumerate}
	
It is important to note that the goal 3(b) is not to enforce isolation, but \remove{merely }to provide isolation for those who require it. As such, apps working together do not automatically violate goal 3(b), but it would be collusion if apps worked together to break isolation with some other non-content app, the system, or the user.

\TODO{many things affect, including non technical}

\TODO{difficult to distinguish}

\TODO{alternative definition \cite{Xu2017}}

\section{Methods for colluding}
\label{sec:methods}

By default, all Android apps run in separate sandboxes\todo{define sandbox}, and by default, all communication between sandboxes is blocked~\cite{AOSP2019b}. However, there are exceptions to both of these statements, as \modif{described below.}\jwnote{through the use of channels.}

\subsection{Overt channels}
\label{sec:overt}

\citeauthor{AOSP2019b} describes channels designed for inter-app communication in \cite{AOSP2019b} and \cite{AOSP2019e}. \jwnote{Move up, it's a general statement for \ref{sec:overt} and \ref{sec:covert}}\discuss{This entire subsection is based on data from these two sources. Is this kind of citation sufficient?}

Apps published by the same \modif{entity}\jwnote{developer} may share a sandbox using the shared UID\jwnote{What is a UID?} feature. In this case, there are no restrictions for their communication. These apps can use any of the traditional UNIX-type mechanisms, including filesystem, local sockets, or signals.

When apps are running in different sandboxes, the Linux kernel\jwnote{SELinux?? Citation for kernel??} prevents \modif{them}\jwnote{apps} from accessing each other's processes or files. In older Android versions, only Linux discretionary access control was used, allowing apps to make their files \modif{world-}\jwnote{globally??}accessible, but newer versions of Android forbid this using SELinux mandatory access control rules. Apps can still use any file-based communication methods when they have permission to access the external storage, but this way users would have some visibility into the fact that such communication channels may be used by the app.

However, Android also provides a method for apps to communicate without any user-granted permission or visibility. This is enabled by a remote procedure call mechanism called binder\jwnote{What is binder? Is this in the Android architecture? I think there needs to be a short introduction to Android platform architecture with figure. See \url{https://developer.android.com/guide/platform}}. Any app can send messages to the binder arbitrarily, but other apps must explicitly start listening and accept incoming communications. \modif{Apps have}\jwnote{The Android platform provides} three main to do \modif{that}\jwnote{this}:
\begin{enumerate}
	\item Services\jwnote{cite}. Apps may start services, which can provide interfaces that are directly accessible using binder.
	\item Intent filters\jwnote{cite}. Intents are simple message objects that represents an "intention" to do something. Apps may ask some part of them to be executed when an intent with certain properties matching their filter is initiated.\jwnote{See \url{https://developer.android.com/guide/components/intents-filters}. Include Figure 1. Explicit/implicit intents. Look at warning boxes, some interesting advice.}
	\item ContentProviders\jwnote{cite}. Apps can define ContentProviders to expose some of their data.
	\JWNOTE{General comment: 1, 2, and 3 are generic. Can they be made more concrete??}
\end{enumerate}

The binder provides an easy way for apps to communicate with each other, promoting openness and allowing separation of concerns. Examples include apps using an intent to ask the camera app to take a photo instead of asking camera control permission, and communication apps allowing other apps to share data through itself. Binder has a well-defined interface that could be monitored. \jwnote{This could go earlier}

\subsection{Covert channels}
\label{sec:covert}

In addition to the overt channels that are designed to be used by apps to communicate, a large amount of covert channels have been described. \citeauthor{Marforio2012}~\cite{Marforio2012} propose a classification of communication channels based on whether Application level APIs, \Gls{os} native calls, or Hardware functionalities are utilised. \citeauthor{Al-Haiqi2014}~\cite{Al-Haiqi2014} describe categorising covert channels as either timing or storage channels. However, neither of these approaches can clearly \remove{and exhaustively} classify all covert channels, which by their nature form an unbounded set. This section provides some examples of covert channels in Android.

\citeauthor{Schlegel2011}~\cite{Schlegel2011} demonstrated that apps can exchange information using the vibration settings. Any application can change the vibration settings without requiring specific permissions. Additionally, applications can subscribe to be notified every time the setting is changed.\jwnote{Is this an intent or a service?} Similarly, the volume setting can be used by apps to exchange information. While apps cannot subscribe to be notified when volume is changed and have to manually check the setting, volume setting has the benefit of having 8 different states, as opposed to the boolean vibration setting. These\jwnote{meaning settings??} channels are invisible to users, as long as data transmission does not coincide with audio playback or receiving notifications.

\jwnote{multiple comments about this paragraph}\citeauthor{Marforio2012}~\cite{Marforio2012} describe how data could be exchanged between\jwnote{colluding} apps by modifying and monitoring the number of threads, processor usage, and free space on filesystem. While \remove{at least} some of the APIs \remove{they} used have \remove{since} been deprecated \cite{nn2017}, partially similar approaches are possible on modern Android versions as well due to their usefulness in legitimate scenarios\tocite{}. For example, free disk space can be queried through a different API on latest Android versions\tocite{}.

The system load can \remove{also} be measured indirectly \remove{in order} to transmit information \cite{Marforio2012}. In this scenario, \jwnote{the}transmitting app modulates data payload by varying the load on the system. \jwnote{the}Receiving app\modif{, on the other hand,}\jwnote{then} repeatedly runs a CPU-intensive computation and measures the time it takes to complete, in order to infer, whether or not the transmitting app is loading the system or not\jwnote{commas}. This approach was \remove{even} shown to work when \modif{receiver is just a}\jwnote{the receiving app is a piece of} JavaScript in the browser and not \modif{a standalone}\jwnote{an installed Android} app.

Another approach is presented by \citeauthor{Al-Haiqi2014}~\cite{Al-Haiqi2014}, where one app utilises the vibration motor to transmit data, and another app uses the accelerometer readings to receive \modif{it}\jwnote{the data}. This is further developed by \citeauthor{}{Qi2018}~\cite{Qi2018}, who propose covert channels based on user behaviour. Instead of using the vibration motor, \jwnote{a}transmitting app could prompt the user to move their phone in certain ways, for example, posing as a rally game where the user needs to turn their phone at specific times based on a track generated by the \jwnote{malicious}app.

\TODO{Add a subsection conclusion paragraph}

\section{Examples of Android app collusion}
\label{sec:examples}

\TODO{Completely rewrite; include a diagram to illustrate}
\sout{A very widely cited example of collusion is an imaginary situation as follows.} \modif{Blah et al. [x] define an example of an Android app collusion as follows:} \JWNOTE{Make the points below a numbered list perhaps. Easier to follow. With a list of numbered steps I would then also think about if a nice diagram could be made, and numbered to correspond to each step described?} One app \modif{, \app{A},} has access sensitive information, but no access to internet. Another app, \modif{\app{B},} \sout{on the other hand,} has access to internet, but no access to any sensitive information. Many authors \tocite{} argue that in this case, one app could pass information to the other one, which could in turn then exfiltrate the information. Some authors \tocite{} have extended this concept to also cover cases where data is passed to multiple apps before being finally exfiltrated. All current research focuses on detecting such situations.

\JWNOTE{The steps are not clear as it is mixed in with discussion. Need to
spend some time separating these.}

There is one known case of Android app collusion in the wild~\cite{Blasco2016}. Interestingly enough, even though this example is also widely referred to \tocite{list some references}, it does not follow the pattern described above. \JWNOTE{This last paragraph is very informal and chatty. This is ok for the draft but would need to be rewritten.} \todo{short description of MoPlus SDK} 

\section{Existing methods for detecting collusions}
\label{sec:approaches}

%============================================================

\nocite{*}
\bibliographystyle{IEEEtranN}
\bibliography{ieee,oro-cs-seminar}
\TODO{format bibliography}
\TODO{remove nocite}

\listoftodos
\TODO{remove list of todos}

\end{document}
