\documentclass[article,oneside]{aaltoseries}
\usepackage[utf8]{inputenc}
\usepackage[english]{babel}
\usepackage{IEEEtrantools}
\usepackage[hidelinks]{hyperref}
\usepackage{amsfonts}
\geometry{left=4cm,right=4cm} % TODO: remove

\begin{document}
\bstctlcite{BSTcontrol}
 
%=========================================================

\title{Android App Collusion}

\author{Ervin Oro% Your first and last name: do _not_ add your student number
\\\textnormal{\texttt{ervin.oro@aalto.fi}}} % Your Aalto e-mail address

\affiliation{\textbf{Tutor}: Jorden Whitefield} % First and last name of your tutor

\maketitle

%==========================================================

\begin{abstract}
Android supports various communication methods between apps, and colluding apps is an emerging threat to Android based devices. An app collusion is where two or more apps collude in some manor to perform a malicious action that an app cannot perform independently. State-of-the-art malware detection systems analyze apps in isolation, and therefore fail to detect joint malicious actions between colluding two or more apps.

\vspace{3mm}
\noindent KEYWORDS: keywords, separated by commas

\end{abstract}


%============================================================


\section{Introduction}

The goal of this seminar topic is to:
\begin{enumerate}
	\item Survey current Android malware detection literature, techniques, and tools.
	\item Survey current Android app collusion literature.
\end{enumerate}


%============================================================

\nocite{*}
\bibliographystyle{IEEEtran}
\bibliography{ieee,oro-cs-seminar}

\end{document}
