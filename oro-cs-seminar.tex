\documentclass[article]{aaltoseries}
\usepackage[utf8]{inputenc}
\usepackage[english]{babel}
\usepackage{IEEEtrantools}
\usepackage[hidelinks]{hyperref}
\usepackage{amsfonts}
\usepackage{todonotes}
\usepackage[normalem]{ulem}
\usepackage{xcolor}
\usepackage{xspace}
\reversemarginpar
\usepackage{textcomp}
\usepackage[numbers]{natbib}
\newcommand{\TODO}[1]{\todo[inline]{#1}}
\newcommand{\jwnote}[1]{\todo[author=jorden,size=\small,inline,color=orange!60]{#1}}
\newcommand{\jwnoteB}[1]{\todo[author=jorden,size=\small,color=orange!60]{#1}}
\newcommand{\modif}[1]{\textcolor{blue}{#1}}
% \geometry{left=4cm,right=4cm,marginparwidth=3.5cm,marginparsep=5mm} % TODO: remove

% Use “\tocite{}” to get “citation needed” in document
\newcommand\tocite[1]{
	\todo{Citation needed: #1}
	[\,\textbf{\color{red}!}\,]
}


% ************************ Custom Referencing **********************************
\newcommand{\Fref}[1]{Figure~\ref{#1}}
\newcommand{\fref}[1]{figure~\ref{#1}}
\newcommand{\Tref}[1]{Table~\ref{#1}}
\newcommand{\tref}[1]{table~\ref{#1}}
\newcommand{\Eref}[1]{Equation~\ref{#1}}
\newcommand{\eref}[1]{equation~\ref{#1}}
\newcommand{\Sref}[1]{Section~\ref{#1}}
\newcommand{\sref}[1]{section~\ref{#1}}
\newcommand{\Lref}[1]{Listing~\ref{#1}}
\newcommand{\lref}[1]{listing~\ref{#1}}


\newcommand{\app}[1]{A\textsc{pp}$_{#1}$\xspace}

\begin{document}

\bstctlcite{BSTcontrol}
 
%=========================================================

\title{Android App Collusion}

\author{Ervin Oro% Your first and last name: do _not_ add your student number
\\\textnormal{\texttt{ervin.oro@aalto.fi}}} % Your Aalto e-mail address

\affiliation{\textbf{Tutor}: Jorden Whitefield} % First and last name of your tutor

\maketitle

%==========================================================

\begin{abstract}
\TODO{abstract}

% \vspace{3mm}
% \noindent KEYWORDS: \todo[inline, inlinewidth=5cm, noinlinepar]{list, of, key, words}

\end{abstract}


%============================================================


\section{Introduction}
\label{sec:intro}
%
\jwnote{Need a more general introduction to mobile computing. Things to mention
would be that there are two leading phone OS's iOS and Android. Focus of this
report is Android. Mention the Android Open Source Project (ASOP) and give an
executive summary of project.}

Android is estimated to be the most widely used operating system \jwnoteB{(OS)}
overall~\cite{AWSLLC2018, StatCounter2018}, running on more than 2 billion
active devices~\cite{AOSP2018}. While this is not reflected by malware,
%
\jwnote{What is not reflected by malware?}
%
most of
which still targets Windows, both the number and complexity of attacks against
Android are increasing~\cite{AVTESTGH2018}.~\citeauthor{McAfee2018} estimates
that revenues for mobile malware authors could be in the billion-dollar range by
2020 \cite{McAfee2018}.

\sout{Taking that into account,}
%
\modif{Given the lucrative business opportunity for cyber criminals,}
%
\sout{devices}
%
defending Android against malware is an active area of research. Thanks to
recent developments,
%
\jwnote{What developments? Research efforts? Don't need to write anything but
add some citations.}
%
exploit pricing and difficulty are growing~\cite{AOSP2018},
indicating that many common attacks
%
\jwnoteB{Add citations to these common attacks. Can you add a category of attack
and a few citations per category?}
%
can now be detected and protected against.
However,
%
\sout{especially with increasing incentives,}
%
malicious actors are looking for
ways to bypass existing protections, and
%
\sout{some}
\modif{a number of}
%
threats,
%
\sout{like}
\modif{e.g.,}
%
app collusion, can not be reliably detected nor defended against.

App collusion is \modif{a} secret collaboration between apps with malicious
intent~\cite{OEDcollusion,Asavoae2017}.
%
\jwnote{citation 6 is strange here. From section two perhaps you could move up
the quote/verbatim of the definition of collusion? Then with this general
definition make it more specific to Android App Collusion.}
%
Android system provides many documented unrestricted channels for apps to
communicate with each other~\cite{AOSP2019b}.
%
\jwnote{What is a channel? What types of channels exist (overt/covert)? I think
it would be good to have a sentence on each channel type, and then add a forward
reference to~\Sref{sec:methods}.}
%
Additionally, apps can use wide
array of covert channels
%
\jwnote{Speak about covert channel here without explaining.}
%
to achieve undetected communication \tocite{}. A
malicious application that would be detected and blocked with state of the art
security systems could be easily split into several applications,
%
\jwnote{What could easily be split across multiple apps? It is not clear.}
%
so that each
of them would be categorized as benign when analysed separately~\cite{Chen2018}.


Android app collusion is not a new concept~\cite{Schlegel2011}, and multiple
attempts have been made to develop a suitable detection system. \TODO{brief
overview of existing approaches}
\jwnote{Mention the figure that literature
does, i.e., problem is {$2^N$}, 2 apps and {$N$} other possibilities.}
% Filtering
% 	rule-based
% 	machine learning
% model based
% runtime

However, there still does not exists any robust and usable ways to detect app
collusions. Most proposed solutions have large number of false positives due to
inability to differentiate collusion from legitimate collaboration.
%
\jwnote{what's the difference between collusion and collaboration? Be explicit
here.}
%
Furthermore,
the only known example of app collusion in the wild~\cite{Blasco2016}
%
\jwnote{Will this papers method be explained prior to making this point? If not
then I don't understand what is out of scope.}
%
would be
out of scope for most current works, as the exponential explosion of having to
analyse all combinations of billions\tocite{number of apps in play store} of
apps has forced authors to focus on a very narrow subset of threats. 
%
\sout{Approaches
attempting to overcome both of these issues have been infeasible thus far.}
\modif{State of the art Android App Collusion detection techniques are limited
in their approach, and thus app collusion remains an open research challenge.}
%

This report aims to provide
%
\sout{a more general}
\modif{an}
%
overview of
%
\sout{the topic.}
\modif{app collusion on the Android platform.}
%
\Sref{sec:def} discusses
%
\sout{the nature of}
%
app collusions in general,~\Sref{sec:methods} provides specific overview of
methods that can be used for colluding on Android,~\Sref{sec:examples}
%
\sout{briefly}
%
describes known examples of colluding apps, and~\Sref{sec:approaches} gives
a more in-depth systematic overview of approaches that have been taken to
collusion detection
%
\sout{, as well as discusses their shortcomings.}
\modif{and their limitations.}


\section{Description and definition of app collusion}
\label{sec:def}
%
The Oxford English Dictionary defines collusion as ``Secret agreement or
understanding for purposes of trickery or fraud; underhand scheming or working
with another; deceit, fraud, trickery.''~\cite{OEDcollusion}.
\jwnote{I think this should be moved up to~\Sref{sec:intro}. This is a general
definition and in section 2 onwards you should be more specific.}

%
\sout{In this context,~\cite{Asavoae2017} defines}
\modif{As{\u{a}}voae et al.~\cite{Asavoae2017} define}
%
app collusion as the situation where several apps are
working together in performing a threat. From this, three
%
\sout{aspects}
\modif{properties}
%
of collusion can be
%
\sout{derived:}
\modif{defined:}
%
\begin{enumerate}
	
	\item Collusion is secret\jwnoteB{I don't understand what you mean. Please
	clarify.}. Conversely, apps working together in
	collaboration is common and encouraged practice when such collaboration is
	well documented \tocite{}.

	\item Collusion is when all colluding parties are in agreement. A distinctly
	different but related concept is the ``confused deputy'' attack, where one
	app mistakenly exposes itself to other installed apps \tocite{Hardy, N.: The
	confused deputy:(or why capabilities might have been invented. ACMSIGOPS
	Operating Systems Review 22(4), 36–38 (1988)}. 

	\item Collusion is with malicious intent. In the Android context, it would
	be the intention to violate one of its security goals,
	%
	\jwnote{What security goals? Should there be a list of these goals earlier
	or in section 1? I also wonder if it is worth mentioning OWASP Mobile top 10
	security... \texttt{https://www.owasp.org/index.php/Mobile\_Top\_10\_2016-Top\_10}}
	%
	which are to protect
	application data, user data, and system resources (including the network)
	\cite{AOSP2019}. The Android Open Source Project also lists providing
	``application isolation from the system, other applications, and from the
	user'' as a separate goal, but in this context, collusion would be when apps
	%
	\sout{work together}
	\modif{collude}
	%
	to break isolation with some third component, as for two
	willing parties there exist several legitimate communication channels
	\cite{AOSP2019b}.
	\jwnote{What is a 3rd component? Is this an app or something else? Are two
	willing parties, two apps?}

\end{enumerate}

\TODO{many things affect, including non technical}

\TODO{difficult to distinguish}

\section{Methods for colluding}
\label{sec:methods}

\TODO{Overt and covert channels}
\jwnote{Anything here about Access Control policies? SEAndroid?}
% \subsection{Overt channels}
% \subsection{Covert channels}

\section{Examples of Android app collusion}
\label{sec:examples}
%
\sout{A very widely cited example of collusion is an imaginary situation as follows.}
\modif{Blah et al. [x] define an example of an Android app collusion as follows:}
\jwnote{Make the points below a numbered list perhaps. Easier to follow. With a
list of numbered steps I would then also think about if a nice diagram could be
made, and numbered to correspond to each step described?}
%
One app
%
\modif{, \app{A},}
%
has access sensitive information, but no access to
internet. Another app,
%
\modif{\app{B},}
\sout{on the other hand,}
%
has access to internet, but no access
to any sensitive information. Many authors \tocite{} argue that in this case,
one app could pass information to the other one, which could in turn then
exfiltrate the information. Some authors \tocite{} have extended this concept to
also cover cases where data is passed to multiple apps before being finally
exfiltrated. All current research focuses on detecting such situations.

\jwnote{The steps are not clear as it is mixed in with discussion. Need to
spend some time separating these.}

There is one known case of Android app collusion in the wild~\cite{Blasco2016}.
Interestingly enough, even though this example is also widely referred to
\tocite{list some references}, it does not follow the pattern described above.
\jwnote{This last paragraph is very informal and chatty. This is ok for the
draft but would need to be rewritten.}
\todo{short description of MoPlus SDK} 

\section{Existing methods for detecting collusions}
\label{sec:approaches}

%============================================================

% \nocite{*}
\bibliographystyle{IEEEtranN}
\bibliography{ieee,oro-cs-seminar}
% \TODO{remove nocite}

% \listoftodos
% \TODO{remove list of todos}

\end{document}
